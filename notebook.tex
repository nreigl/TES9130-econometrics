\documentclass{article}\usepackage[]{graphicx}\usepackage[]{color}
%% maxwidth is the original width if it is less than linewidth
%% otherwise use linewidth (to make sure the graphics do not exceed the margin)
\makeatletter
\def\maxwidth{ %
  \ifdim\Gin@nat@width>\linewidth
    \linewidth
  \else
    \Gin@nat@width
  \fi
}
\makeatother

\definecolor{fgcolor}{rgb}{0.345, 0.345, 0.345}
\newcommand{\hlnum}[1]{\textcolor[rgb]{0.686,0.059,0.569}{#1}}%
\newcommand{\hlstr}[1]{\textcolor[rgb]{0.192,0.494,0.8}{#1}}%
\newcommand{\hlcom}[1]{\textcolor[rgb]{0.678,0.584,0.686}{\textit{#1}}}%
\newcommand{\hlopt}[1]{\textcolor[rgb]{0,0,0}{#1}}%
\newcommand{\hlstd}[1]{\textcolor[rgb]{0.345,0.345,0.345}{#1}}%
\newcommand{\hlkwa}[1]{\textcolor[rgb]{0.161,0.373,0.58}{\textbf{#1}}}%
\newcommand{\hlkwb}[1]{\textcolor[rgb]{0.69,0.353,0.396}{#1}}%
\newcommand{\hlkwc}[1]{\textcolor[rgb]{0.333,0.667,0.333}{#1}}%
\newcommand{\hlkwd}[1]{\textcolor[rgb]{0.737,0.353,0.396}{\textbf{#1}}}%

\usepackage{framed}
\makeatletter
\newenvironment{kframe}{%
 \def\at@end@of@kframe{}%
 \ifinner\ifhmode%
  \def\at@end@of@kframe{\end{minipage}}%
  \begin{minipage}{\columnwidth}%
 \fi\fi%
 \def\FrameCommand##1{\hskip\@totalleftmargin \hskip-\fboxsep
 \colorbox{shadecolor}{##1}\hskip-\fboxsep
     % There is no \\@totalrightmargin, so:
     \hskip-\linewidth \hskip-\@totalleftmargin \hskip\columnwidth}%
 \MakeFramed {\advance\hsize-\width
   \@totalleftmargin\z@ \linewidth\hsize
   \@setminipage}}%
 {\par\unskip\endMakeFramed%
 \at@end@of@kframe}
\makeatother

\definecolor{shadecolor}{rgb}{.97, .97, .97}
\definecolor{messagecolor}{rgb}{0, 0, 0}
\definecolor{warningcolor}{rgb}{1, 0, 1}
\definecolor{errorcolor}{rgb}{1, 0, 0}
\newenvironment{knitrout}{}{} % an empty environment to be redefined in TeX

\usepackage{alltt}
% \usepackage{amsfonts}
    \usepackage{amsmath}
    \usepackage{amssymb}

    \usepackage[splitrule]{footmisc}
    \interfootnotelinepenalty=10000 %% Completely prevent breaking of footnotes
    
    % \usepackage{mathpazo}
    % \usepackage{hyperref\}
    % \usepackage{subfigure}
    % \usepackage{tikz}
\usepackage[authoryear,round,longnamesfirst]{natbib}

 \usepackage{multirow}
    \usepackage{graphicx}



 \usepackage{tikz}
 \usepackage{listings}
 \usepackage{nth}  % proper superscripts with \nth{1}, \nth2 , etc
 % Table formatting
    \usepackage{tabularx}
    \usepackage{ltablex}
\IfFileExists{upquote.sty}{\usepackage{upquote}}{}
\begin{document}

fourth unit: 

$E(\epsilon |X )= \sigma^2 $


$P$ is also a $n \times n $ matrix

feasible GLS where we estimate the $P$ and the $\Psi$

1) Assignment
applied articles where th problem has been addressed empirically. plus solutions based on theoretical work 

2) no STATA comments. theory: contextual issues. which assumptions is strongest. conceptual understanding. 

 
 $C_{1}=(1+r)^{-\sigma} \beta^{-\sigma} [(1+r)(Y_{1}-C_{1})+Y_{2}] $


$C_{1} = {\frac{1}{1+(1+r)^{\sigma -1} \beta ^\sigma}} [Y_{1}+ \frac{Y_{2}}{1+r}]$


$\mathbb{E}(x) = \alpha_{0}+\alpha_{1} x^{*}+\alpha_{2}z $
and 
$\mathbb{E}(y) = \beta_{0}+\beta_{1} x+f(c) $


where $ x^{*} is some latent part of $x$ and $c$ is still unobserved

%  1) Assignment No 1 (20%): Empirical approach to the research problem or research question chosen by the student. Write an overview of the econometric approach (based on existing literature) applied regarding the chosen research question. Elaborate on the challenges and proposed solutions regarding the econometric analysis of the research problem. Make use of the most cited articles (refer to the list of econometric articles provided in Moodle) which deal with the econometric problems and provide solutions to them and which relate to chosen research question. Students are expected to exchange their papers and to comment on other students’ methodological overview of the research problem (distribution of papers takes place in Moodle Workshop). Each student is expected to make a 15 minutes presentation on its assignment and give comments (5-8 minutes) to the peer student paper.



\end{document}
